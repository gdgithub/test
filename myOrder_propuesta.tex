\documentclass[12pt]{article}

\usepackage[spanish]{babel}
\usepackage[utf8]{inputenc}

\begin{document}
\title{MyOrders}
\date{\today}
\author{Starlin Francisco Gil Cruz\\
Ivan Francisco Gil Cruz}
\maketitle

\begin{abstract}
En este documento se presenta la propuesta de la aplicación y los detalles tecnicos relativos al desarrollo de la misma, asi como 
la manera de ejecutarla en un ambiente de producción.
\end{abstract}

\section{Propuesta}
La aplicación basicamente es un sistema de ordenes que permite llevar el control de los pedidos que los usuarios realizan, por medio de la aplicación, a los distintos establecimientos de alimentos.\\

En la pagina principal de la aplicación se tiene la opcion para registrar un nuevo usuario e iniciar la sesion de uno existente. Cuando este inicia sesion se guarda en las cookies las credenciales para no tener que volver a iniciar la sesion la proxima vez de utilizarse. Ademas de esto, se cuenta con la posibilidad de el directorio desde la pagina principal; en el cual aparecen los establecimientos de comida clasificados por su categoria y ordenados por el rating de manera descendente. Los usuarios que no han iniciado sesion pueden ver el directorio, pero no hacer pedidos a los establecimientos.\\ 

La aplicación maneja tres niveles de usuarios: developer, firefighter y administrador. El rol developer es el que todos los usuarios que se registran desde la aplicación poseen; con este pueden realizarse los pedidos a establecimientos, una vez que el usuario a completado su datos personales como: dirección, etc, ver el grupo y los integrante del grupo al cual pertenecen, en caso de pertenecer algun grupo, y crear sugerencias de establecimientos. El administrador decide si la sugerencia se aplicara o no.\\

Los usuarios con el rol firefighter pueden hacer las mismas acciones que los usuarios developer, con la excepcion que pueden crear grupos y agregar miembros a estos. Un miembro no puede aparecer en dos grupos diferentes.


\section{Aplicacion}
La aplicación funciona en un ambiente web tanto para desktop como para dispositivos moviles; ademas de que se implementa la aplicación para Android. La Api de Android que utilizamos es la version 23.\\ 

Por falta de tiempo no pudimos realizar la aplicacion Android en esta fase.

\section{Datos de acceso}

\begin{tabular}{l c c}
Usuario & Contraseña & Rol\\ \hline
starlin.gil.cruz@gmail.com & 1234 & Admin\\
ivangilcruz@gmail.com & 1234 & FireFighter\\
dev1@gmail.com & 1234 & Developer\\
dev2@gmail.com & 1234 & Developer
\end{tabular}




\end{document}
